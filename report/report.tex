\documentclass{article}
\usepackage[utf8]{inputenc}

\usepackage{sectsty}
\usepackage{multicol}
\usepackage{amsmath}
\usepackage{listings}
\usepackage{booktabs}
\usepackage{siunitx}  
\usepackage[margin=0.5in]{geometry}
 
\sectionfont{\fontsize{12}{15}\selectfont}

\title{\vspace{-2.0cm}\textbf{PageRank: A Parallel Approach}}
\author{Manos Chatzakis\\csd4238@csd.uoc.gr\\ \\ Computer Science Department \\ University Of Crete}
\date{March 2021}

\begin{document}

\maketitle

%\begin{abstract}
%There are plenty of approaches that try to parallel the PageRank algorithm efficiently. This report %presents a method to run concurrently the PageRank algorithm on a directed graph, without the use of %locks, by dividing the input graph is a selected number of threads.
%\end{abstract}

\section*{1. Introduction}
This report is part of the first assignment of cs342 - Parallel Programming course of Computer Science Department, University of Crete. The main point is to present a method to parallel PageRank algorithm efficiently, and the rest of the report is organized as follows: Section 1 is the introduction. Section 2 describes the input graph.
Section 3 analyzes the parallelism method. Section 4 shows the measurements. Section 5 is the conclusion.

\section*{2. Input Graph}
\subsection*{2.1 Structure}
The input is converted into a PageRank graph:
\begin{multicols}{3}
\lstinputlisting[language=c]{g_s.c}
\lstinputlisting[language=c]{n_s.c}
\lstinputlisting[language=c]{l_s.c}
\end{multicols}

\subsection*{2.2 Graph Division}
For the work division, every thread handles a set of nodes, determined from the equation:
 \[sets = \frac{graph.size}{threads}\]
The above method requires less work spend on the data parsing, but it does not always result to the best work division. Another approach could try to divide the nodes in a way that nodes with high work demand (etc. many outlinks or inclinks) are given to different threads, but it would result in high overhead for dataset manipulation.

\section*{3. Algorithm and Approach}
\subsection*{3.1 Formula}
Our approach calculates the PageRank of each node using the simplified formula:\\

%\[ PR(u) = \sum_{v\in Bu}^{}DF\times \frac{PR(v)}{L(v)}\] \\

\begin{equation}
    PR_n(u) = 
    \begin{cases}
      \displaystyle\sum_{k\in B_u}^{}DF \frac{PR(k)}{L(k)} + PR(u), & \text{if}\ OutDegree(u) = 0 \\\\
      \displaystyle\sum_{k\in B_u}^{}DF \frac{PR(k)}{L(k)} + (1 - DF)\times PR(u), &  \text{if}\ OutDegree(u) > 0
    \end{cases}
\end{equation}
\[\Rightarrow\]
\begin{equation}
   PR_n(u) = PR(u) + \sum_{k\in B_u}^{}DF \frac{PR(k)}{L(k)} - \sum_{k\in T_u}^{}DF \frac{PR(u)}{L(u)}
\end{equation}
\\\\
PR(x): PageRank of node x in the current iteration,
PRn(x) : PageRank of node x for the next iteration,
L(x): The number of outgoing links of node x,
Bu: Set containing the nodes that point node u,
Tu: Set containing the nodes that u points at,
DF: Dumping factor, e.g. the percentage that every node offers from its own PageRank. In this approach, this value is considered 0.85, thus every node offers 85\% of his PageRank in every iteration. 
\\
The above formula implies that the PageRank of a node in the next iteration is the sum of all the scores given from the neighbors, incremented with the previous score minus the score that this node gave throughout the iteration. Note that using this equation means that the total PageRank score of the graph remains constant.

\subsection*{3.2 Parallelism}
The parallelism method relies on the observation that PRn(u) refers to the PageRank value that will be used in the next iteration. Thus, this implementation saves the offered PageRank score by each neighbour to a new field of the node struct, and adds it to the PageRank score when the calculations of the corresponding iteration end. This method does not face data race, as only the thread assigned the node has read/write access to the temporary node field:

\lstinputlisting[language=c]{calc.c}

An important part for this implementation is the use of barriers, to ensure that the score update for each node and the start of the new iteration for each thread will be done concurrently.

\lstinputlisting[language=c]{barriers.c}

\section*{4. Measurements}
In this section the different approaches are measured and the metrics are presented. The values below were extracted using 3 SNAP datasets (Facebook, Gnutella, Enron) on a local machine (8 cores,16 GB RAM), using the command:
\lstinputlisting[language=c]{clock.c}
The time is measured in seconds.
\begin{itemize}
  \item Facebook Dataset.
 \begin{center}
    \begin{tabular}{||c | c c c c c | c c c | c||} 
    \hline
    Threads & Run 1 & Run 2 & Run 3 & Run 4 & Run 5 & Average & Standard Dev. & Average Dev. & Speedup* \\ [0.5ex] 
    \hline\hline
    1 & 0.038759 & 0.037749 & 0.039297 & 0.04045 & 0.039666 & 0.03918 & 0.00101 & 0.00074 & - \\ 
    \hline
    2 & 0.021782 & 0.02137 & 0.02144 & 0.021435 & 0.022365 & 0.2168 & 0.00042 & 0.00032 & 1.807x \\ 
    \hline
    3 & 0.015493 & 0.01498 & 0.015683 & 0.015355 & 0.015551 & 0.01541 & 0.00027 & 0.0002 & 2.542x \\ 
    \hline
    4 & 0.015658 & 0.015022 & 0.014813 & 0.014786 & 0.014616 & 0.01498 & 0.00041 & 0.00029 & 2.615x \\ 
    \hline
    \end{tabular}
    \end{center}
  
  \item Gnutella Dataset.
  \begin{center}
    \begin{tabular}{||c | c c c c c | c c c | c||} 
    \hline
    Threads & Run 1 & Run 2 & Run 3 & Run 4 & Run 5 & Average & Standard Dev. & Average Dev. & Speedup* \\ [0.5ex] 
    \hline\hline
    1 & 0.056732 & 0.055963 & 0.059056 & 0.057208 & 0.06052 & 0.0579 & 0.00186 & 0.00151 & - \\ 
    \hline
    2 & 0.038491 & 0.037579 & 0.036734 & 0.038122 & 0.038693 & 0.03792 & 0.00079 & 0.00061 & 1.526x \\ 
    \hline
    3 & 0.027344 & 0.028106 & 0.028176 & 0.027397 & 0.027783 & 0.02776 & 0.00039 & 0.00031 & 2.085x \\ 
    \hline
    4 & 0.023385 & 0.022163 & 0.022034 & 0.025295 & 0.023941 & 0.2336 & 0.000135 & 0.00101 & 2.478x \\ 
    \hline
    \end{tabular}
    \end{center}
    
    \item Enron Dataset.
    \begin{center}
    \begin{tabular}{||c | c c c c c | c c c | c||} 
    \hline
    Threads & Run 1 & Run 2 & Run 3 & Run 4 & Run 5 & Average & Standard Dev. & Average Dev. & Speedup* \\ [0.5ex] 
    \hline\hline
    1 & 0.642836 & 0.649105 & 0.617986 & 0.666572 & 0.667904 & 0.64888 & 0.02041 & 0.01478 & - \\ 
    \hline
    2 & 0.590239 & 0.588571 & 0.588146 & 0.569202 & 0.583949 & 0.58402 & 0.0086 & 0.00596 & 1.111x \\ 
    \hline
    3 & 0.513008 & 0.513008 & 0.515894 & 0.511074 & 0.579348 & 0.52647 & 0.02961 & 0.02115 & 1.232x \\ 
    \hline
    4 & 0.482256 & 0.484116 & 0.479822 & 0.54109 & 0.492112 & 0.49588 & 0.02569 & 0.01808 & 1.308x \\ 
    \hline
    \end{tabular}
    \end{center}
  
\end{itemize}

\\Measurements on a CSD machine are also provided. (*): Speedup was calculated using the formula:
\[SpeedUp = \frac{t_s_e_q}{t_p_a_r} = \frac{thread1_a_v_e_r_a_g_e}{threadN_a_v_e_r_a_g_e} = S \Rightarrow SpeedUp = S x \]\\
\[Assuming: t_s_e_q = thread1_a_v_e_r_a_g_e\]



\section*{5. Conclusion}
The approach presented seems to improve the running time of PageRank, especially when the multi-threaded program runs on 2 or 3 threads. Thus, PageRank can be very efficient, even when its used for big graphs.

\end{document}
